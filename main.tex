\documentclass[12pt]{article}
\usepackage[utf8x]{inputenc}
\usepackage[spanish]{babel}
\usepackage{url}
\usepackage{lipsum}
\usepackage{graphicx}
\usepackage{color}
\usepackage{amsmath}
\usepackage{amssymb}
\usepackage{hyperref}
\usepackage{listings}
\usepackage{xcolor}
\usepackage{float} % here for H placement parameter
\usepackage{tabularx}


\title{Propuesta de Tesina}

\begin{document}
\author{Lucio Bassani}

\maketitle

\begin{figure}[h]
\centering
\includegraphics[width=4in]{LOGO-UNR-NEGRO.png}
\end{figure}

\textbf{Postulate:} Lucio Bassani.

\textbf{Director:} Alejandro Sartorio.

% agrego comentario para commitear y ver el sync


\section{Situación del Postulante}
El postulante aprobó todas las materias. Se encuentra trabajando con una carga semanal de 40 horas.

\section{Título}
Una infraestructura técnica y procedimental para el análisis de noticias de portales digitales a través de inteligencia artificial generativa.

% Previo: Análisis de Datos, Redes Sociales y Generación de Contenido por IA en Campañas Políticas
% Ale: Una infraestructura técnica y procedimental para el análisis de noticias de portales digitales a través de inteligencia artifical generativa


\section{Motivación y Objetivo General}\label{sec:Moti_O_G}

% - Referencia sobre lo que hay que escribir - Explique el problema o situaci´on de referencia en el que se desarrolla la propuesta o los nterrogantes en el campo disciplinario a los que la propuesta se dirige. Desarrolle la importancia e impacto de los objetivos y el conocimiento que se generar´a. En esta secci´on no es necesario  escribir las tareas espec´ıficas que se realizar´an (para eso, ver Objetivos espec´ıficos).

El panorama informativo actual se caracteriza por la proliferación de portales digitales, que ofrecen una amplia variedad de noticias en línea. Este aumento de datos e información ha dificultado el seguimiento y el análisis de las noticias, lo que ha limitado la capacidad de los investigadores y los profesionales de la información para determinar y comprender su impacto en la sociedad.

Para superar este desafío, se necesitan instrumentos de análisis de texto más sofisticados que permitan identificar correlaciones entre noticias, acoplamiento y cohesión de los términos, conceptos y significantes. Estos instrumentos proporcionarán a los investigadores un sólido respaldo tecnológico para automatizar y optimizar sus protocolos de estudio, incluyendo la formación de la agenda mediática y la difusión de información de manera más eficiente.


Una de las áreas de aplicación más relevantes de estos estudios se centra en la influencia de los medios de comunicación tradicionales en la configuración de la agenda pública, así como en el creciente impacto de las redes sociales como caja de resonancia de noticias y su llegada a públicos segmentados. Desde la publicación del influyente artículo ``La función de establecer la agenda de los medios de comunicación de masas" (McCombs, 1972)\cite{mccombs:1972}, se ha reconocido el papel fundamental de los medios de comunicación en la configuración de la percepción pública. No obstante, en la última década, la irrupción de las redes sociales, primero con la blogosfera y posteriormente con los sitios de aquellas, ha alterado significativamente este panorama.

% Ale - ok 

El desarrollo de las mediciones y la obtención de indicadores referidos a noticias publicadas en medios digitales se fue consolidando en los últimos tiempos. Para este propósito se utilizan técnicas de análisis y \textbf{procesamiento de lenguaje natural} (NLP, por sus siglas en inglés de ``Natural Language Processing"); \textbf{ciencia de datos} para la identificación de entidades, relaciones e indicadores claves (KPI, por sus siglas en inglés de ``Key Performance Indicator''); \textbf{clasificación} mediante redes neuronales, etc. 

Siguiendo con estos lineamientos, se plantea la hipótesis de la utilización de los servicios de inteligencia artificial generativa (en adelante, IAG) en un proceso original, combinados con técnicas de ciencias de datos y análisis de texto para organizar el análisis de noticias y representación de los resultados.
Para su implementación se utilizarán los siguientes datos de entrada para la construcción de los "prompts" de la IAG a utilizar: 

    \begin{enumerate}
        \item Los datos empíricos de las noticias con el agregado de información de contexto. Ejemplo:  tipo de portal, georeferencia, importancia de la noticia, ubicación, etc.
        \item Los objetivos requeridos para el análisis. Ejemplo: alcance, paradigma de análisis, referencias a tener en cuenta, etc.
        \item El tipo de salida o representación del análisis. Ejemplo: narrativa, tablas, gráficos, etc.
    \end{enumerate}

%Mismo comentario?
% Ale propone sacar esto -  En esta tesina se plantea un desafío crítico: la necesidad de acceder y analizar información proveniente de portales digitales. Se tendrá en cuenta la posibilidad de captar la mayor cantidad posible de variedad de portales, permitiendo una clasificación según la influencia que pueda tener en los resultados de los análisis. 
% Ale - reemplazar párrafo: En esta tesina se plantea un desafío crítico: la necesidad de acceder y analizar información proveniente de portales digitales. Se tendrá en cuenta la posibilidad de capatar la mayor cantidad posible de variedad de portales, permitiendo una clasificación según la influencia que pueda tener en los resultados de los análisis.

% Ale propone sacar este párrafo completo - El objetivo fundamental de esta tesina es diseñar y automatizar un proceso que facilite la obtención y el procesamiento de noticias vinculadas a la elección presidencial, seguido de la generación de información susceptible de análisis mediante técnicas de procesamiento de lenguaje natural (NLP) y análisis de datos. 
% Dentro de nuestra propuesta se incluye la creación de una herramienta que permitirá la personalización de "prompts" para inteligencias artificiales generativas, simplificando así la generación de contenido específico basado en los datos recopilados.

Teniendo en cuenta las necesidades e hipótesis de solución se define un objetivo general para este trabajo de la siguiente manera: 

\begin{quote}
        \textbf{Crear un framework para automatizar el análisis de noticias digitales a través de inteligencia artificial generativa.
        }
\end{quote}

El cumplimiento de los objetivos planteados en esta propuesta de tesis tendría los siguientes beneficios:


\begin{enumerate}
    \item \textbf{Aumento de la eficiencia del análisis:} la automatización del análisis permitiría a los usuarios ahorrar tiempo y recursos. 

    \item \textbf{Capacidad de adaptabilidad al contexto:} la IAG es una tecnología que ha demostrado ser capaz de generar resultados según información de contexto, permitiendo una alta flexibilidad de los resultados según diferentes escenarios, alcance y dominio de aplicación. 

    \item \textbf{Ampliación del alcance del análisis:} dado que las IAG pueden procesar grandes cantidades de datos de forma rápida y eficiente sería posible analizar un mayor número de noticias.

\end{enumerate}


\section{Fundamentos y estado de conocimiento sobre el tema}

% Ale deja estas indicaciones: 
% Podemos ir organizando el estado del arter de forma de contar cuales son los trabajos que reflexionan o hacen aportes tecnológicos sobre los conceptos y piezas que nosotros vamos a necesitar para esta tesis. 
%To measure is one way of performing an analysis. “What gets counted counts”, claim Klein and
%


Según Luhman (2000, p. 1) \cite{luhmann2000}: ``\textit{Todo lo que sabemos acerca de nuestra sociedad, o más aún, acerca del mundo en el que vivimos, lo conocemos a través de los medios de comunicación masiva.}''. Es por esto que el estudio de la influencia de los medios de comunicación en la opinión pública ha sido un campo de investigación activo durante décadas, en el que se han desarrollado diferentes tipos de tecnologías, metodologías y métricas para medir este fenómeno. Entre las tecnologías más utilizadas se encuentran los sondeos de opinión, los análisis de contenido y las técnicas de neurociencia. Un ejemplo concreto de como se han utilizado estas herramientas es el estudio de la influencia de la cobertura mediática de los debates electorales en las preferencias de los votantes. 

% Ale - Se puede poner estas referencias:
%Groseclose, T., & Milyo, J. (2005). A measure of media bias. Quarterly Journal of Economics, 120(4), 1191-1237.
%Iyengar, S., & Kinder, D. R. (1987). News that matters: Television and American opinion. Chicago, IL: University of Chicago Press.
%Price, V., & Tewksbury, D. (1997). News values and public opinion: A new look at agenda-setting. Communication Research, 24(5), 577-594.


% ver todas estas referencias para citarlas acá y a lo largo de esta sección: https://scholar.google.com/scholar?hl=es&as_sdt=0%2C5&q=El+estudio+de+la+influencia+de+los+medios+de+comunicaci%C3%B3n+en+la+opini%C3%B3n+p%C3%BAblica+&btnG=

En cuestiones de análisis de datos a través de mediciones en portales de noticias online se pueden mencionar los siguientes trabajos:

La tesis doctoral de Pablo Rey-Mazón (2023), ``Color of Corruption'' \cite{mazon:2023} presenta varias herramientas y una metodología novedosas para medir el proceso de \textit{agenda-setting} en un ecosistema mediático complejo.

Según Rey-Mazón la metodología tradicional de análisis de contenido se basa en el supuesto de que el contenido ya está completo y estático para su estudio. Este método presenta desafíos, ya que no toma en cuenta el flujo constante de noticias y su posible influencia en cómo los lectores reciben la información. En este contexto, Rey-Mazón presenta el software \textbf{Homepagex} que sigue la posición de los elementos de noticias en la página de inicio durante varios días, lo que permite obtener una imagen completa de los flujos y ciclos de noticias en los sitios web. Este software combina la construcción de una base de datos -mediante el rastreo web de las páginas de inicio a cada hora- con el análisis de contenido automatizado, lo que ofrece una posible forma de abordar el análisis de datos en el estudio actual.

Otra herramienta planteada, denominada ``Color of Corruption'', consiste en una base de datos de imágenes de noticias sobre corrupción. La base de datos está etiquetada con una serie de atributos, como el tono emocional, el enfoque narrativo y la presencia de actores específicos. La metodología, denominada \textit{``Visual Agenda-Setting Analysis''}, utiliza esta herramienta para analizar la cobertura mediática de los escándalos de corrupción. La tesis de Rey-Mazón es relevante para la presente investigación, ya que presenta una herramienta y una metodología de trabajo con características similares a la que se pretende construir en este trabajo. 

% Todo Lucio:  Sacar información de la página 22 de la tesis "el color de la corrupción" 

% Lucio: deberia agregar mas metodos de analisis?

Otras de las temáticas de estudio que se propone, se relaciona con el uso de algoritmos de inteligencia artificial generativa a través de los servicios basados en modelos similares a:  


\begin{itemize}
    \item LaMDA (Language Model for Dialogue Applications): modelo de lenguaje generativo de Google AI que se puede utilizar para crear nuevas noticias, traducir idiomas, escribir diferentes tipos de contenido creativo, y responder a preguntas de forma informativa.
    \item DALL-E 2: modelo de generación de imágenes de OpenAI que se puede utilizar para crear imágenes a partir de texto, lo que puede ser útil para ilustrar noticias periodísticas.
    \item GPT-3, GPT-J, GPT-Neo, GPT-NeoX: son modelos de lenguaje generativo de OpenAI que se puede utilizar para generar nuevos textos, lo que puede ser útil para crear nuevas noticias o traducir idiomas.
    \item Meena: modelo de lenguaje generativo de Google AI que se puede utilizar para generar conversaciones de chat realistas.
    \item Bard: modelo de lenguaje generativo de Google AI que se puede utilizar para generar diferentes formatos de texto creativo, como poemas, código, guiones, piezas musicales, correo electrónico, cartas, etc.
    \item GPT-4: modelo de lenguaje generativo de OpenAI que aún está en desarrollo, pero se espera que sea aun más avanzado que los modelos anteriores. 
\end{itemize}

También será necesario contar con información de base que permita el diseño e implementación de procesos tomando como referencia antecedentes de construcción de flujos de trabajos guiados por estándares. Este tipo de construcciones son tareas complejas que requieren la participación de expertos provenientes de diferentes áreas. 

Una referencia relevante en este campo es el trabajo de Prades et al. (2013) \cite{PRADES2013115}, que define una metodología para el diseño e implementación de modelos de procesos de negocio en BPMN, de acuerdo con el estándar ANSI/ISA-95. Este trabajo proporciona una guía detallada para la construcción de procesos en entornos de fabricación.

Además de la información de base, es necesario contar con aplicaciones de procesos que incorporen servicios de inteligencia artificial generativa. Un ejemplo de la aplicación de IAGs en procesos es el trabajo de Bertagnolio et al. (2023) \cite{bertagnolio2023}, que utiliza un modelo de aprendizaje automático para generar códigos de ruido de turbinas eólicas. Este modelo se utiliza para comparar y verificar diferentes códigos de ruido.


\begin{enumerate}
\item \textbf{Influencia de las Redes Sociales en la política}: se ha investigado extensamente cómo las redes sociales influyen en la percepción pública, la participación ciudadana y la toma de decisiones políticas. Estudios han demostrado el impacto de plataformas como Twitter y Facebook en la difusión de información política y la formación de opiniones.

\item \textbf{Análisis de sentimiento y temas}: las técnicas de análisis de sentimiento se han utilizado para evaluar la actitud del público hacia candidatos y temas políticos en las redes sociales. Esto permite comprender cómo las conversaciones en línea pueden influir en las estrategias de campaña y la percepción pública.

\item \textbf{Generación de Contenido por IA}: la generación de contenido por medio de IA, incluyendo el uso de modelos de lenguaje generativo como GPT-3, ha abierto nuevas posibilidades para crear mensajes y discursos políticos personalizados. Esto es relevante tanto para la comunicación de campaña como para la interacción en redes sociales.

\item \textbf{Modelado de procesos tecnológicos}: la aplicación de procesos tecnológicos, como el modelado de flujos de trabajo (BPMN), para automatizar la recopilación y análisis de datos en campañas políticas ha ganado atención. Estos procesos pueden optimizar la toma de decisiones y la asignación de recursos en tiempo real.

\item \textbf{Casos de estudio y aplicaciones prácticas}: investigaciones previas han analizado elecciones específicas y campañas políticas para demostrar la eficacia de enfoques tecnológicos y de análisis de datos en la mejora de estrategias de comunicación y toma de decisiones.
\end{enumerate}

%En resumen, la intersección entre análisis de datos, redes sociales, generación de contenido por IA y campañas políticas es un campo de investigación en crecimiento que tiene el potencial de transformar la manera en que se conciben y ejecutan las estrategias políticas. La comprensión de estos fundamentos y la exploración del estado del conocimiento existente proporcionará una base sólida para esta investigación.

\section{Objetivos específicos}

Para la concreción del objetivo general propuesto en la sección \ref{sec:Moti_O_G} se plantean los siguientes objetivos específicos. 

    \begin{quote}
    \begin{itemize}
        \item [OE1:] \textbf{Desarrollar una aplicación informática que permita gestionar el proceso integral de extracción y transformación de datos, análisis y muestra de resultados.}
    \end{itemize}
    \end{quote}

        \vspace{0.3 cm}

        El alcance de este objetivo es crear un módulo específico que pueda ser integrado en la suite del framework denominado \textit{Odoo  fundation} \footnote{https://www.odoo.com/es\_ES} en su versión comunitaria de código abierto.
        El módulo contará con servicios gestión para los valores de parametrización de los algoritmos que intervienen en el proceso que se implementa. El usuario final podrá cargar, a través de formularios, la información necesarias para el acceso a los portales de noticias, las reglas de búsquedas de información y criterios sobre la aplicación de procedimientos. Se desarrollarán integraciones con otros módulos de la suite de Odoo que permitan la visualización de la información de resultados en tableros, informes y reportes adecuados para el propósito de aplicación.

        \vspace{0.3 cm}

        \begin{quote}
            \begin{itemize}
                \item [OE2:] \textbf{Diseñar e implementar, en la aplicación informática, un proceso que organice las tareas y procedimientos de las etapas de recolección de datos, análisis y visualización.} 
             \end{itemize}
        \end{quote}

        \vspace{0.3 cm}

        En este objetivo específico, se pretende organizar y automatizar la hoja de ruta necesaria para satisfacer a los expertos del dominio y usuarios finales de la tecnología. Las tareas que se deben ejecutar forman parte de un proceso que debe ser diseñado e implementado tecnológicamente.

        
        \vspace{0.3 cm}

        \begin{quote}
            \begin{itemize}
                    \item [OE3:] \textbf{Diseñar e implementar los procedimientos de acceso a los servicios de IAG guiado por "prompts".}
            \end{itemize}
        \end{quote}

        \vspace{0.3 cm}

        Los servicios de inteligencia artificial generativa que se utilizarán en las tareas específicas del proceso, serán diseñados a través de un estilo de arquitectura que permita una adecuada integración al proceso mencionado. Para este propósito, se utilizarán los fundamentos del estilo de arquitectura denominada Tubos y Filtros \cite{land2002brief}
        

\section{Metodología y Plan de Trabajo}

La metodología propuesta para llevar adelante este trabajo de tesina se basa en un enfoque iterativo e incremental. Esto significa que el trabajo se desarrollará en ciclos cortos, que se centrarán, cada uno, en un objetivo específico. Al final de cada ciclo, se realizará una revisión periódica para evaluar los avances y evolución del material de estudio, productos consolidados y alineación a un plan de trabajo consensuado con la dirección.

La revisión periódica es una actividad esencial para garantizar la calidad del trabajo y el cumplimiento de los objetivos definidos. En esta, se evaluarán los siguientes aspectos:

- Se comprobará que el material de estudio este actualizado y que cubra los aspectos necesarios para el desarrollo de la tesis.

- Se revisarán los productos consolidados, como documentos, prototipos o software, para garantizar su calidad y funcionalidad.

- Se comprobará que el trabajo este alineado con el plan director de trabajo consensuado con la dirección.


\vspace{1cm}

Se prevé asignar un período de estudio diario de 2 a 3 horas para el desarrollo de la tesina. A continuación, se detallará de manera más exhaustiva cómo se ejecutarán los aspectos mencionados en la sección previa:

\begin{enumerate}
    \item El proceso se iniciará con la revisión de la literatura pertinente: papers, publicaciones y presentaciones audiovisuales afines.
    \item Se desarrolla la aplicación para extracción, transformación y guardado de datos en el software de gestión Odoo.
    \item Diseño de proceso organizativo de tareas de las etapas de recolección, análisis y visualización de datos como procesos de negocio en BPMN.
    \item Selección de modelo de inteligencia artificial generativa y los servicios de accesos para su implementación.
    \item Diseño de interfaces para acceder al modelo seleccionado.
    \item Implementación de interfaz y experiencias de usuarios para la gestión de ``prompt'' a usar en el modelo.
    \item Desarrollos informáticos 
    \item Recopilación de datos y generación de conclusiones.
    \item Escritura de informe.
\end{enumerate}


\subsection{Cronograma de trabajo}

En la siguiente tabla se detalla el cronograma de tareas y entregables definido para el cumplimiento de los objetivos establecidos. 


\begin{table}[h]
\centering{
\begin{tabular}{|c|l|c|c|c|c|c|c|}
\hline
\textbf{Tarea} & \textbf{Entrega} & \multicolumn{6}{c|}{\textbf{Meses}} \\
\hline
 1 & Papers categorizados & x & & & & & \\
\hline
 2 & Aplicación de scrapeo & & x & x & & & \\
\hline
 3 & Gráfico de diseño &  &  & x & & & \\
\hline
 4 & Modelo seleccionado &  &  & x & & & \\
\hline
 5 & Guía de interfaces &  &  & x & x & & \\
\hline
 6 & Aplicación de prompt & & & x &x  & & \\
\hline
 7 & Módulo Odoo &  &  & &x &x & \\
\hline
 8 & Capítulo final & & & & &x & \\
\hline
 9 & Informe &  &  & & &x &x \\
\hline
\end{tabular}
}
\end{table}





\bibliographystyle{unsrt} % We choose the "plain" reference style
\bibliography{refs} % Entries are in the refs.bib file
\end{document}
